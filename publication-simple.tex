\documentclass[]{article}
\usepackage{lmodern}
\usepackage{amssymb,amsmath}
\usepackage{ifxetex,ifluatex}
\usepackage{fixltx2e} % provides \textsubscript
\ifnum 0\ifxetex 1\fi\ifluatex 1\fi=0 % if pdftex
  \usepackage[T1]{fontenc}
  \usepackage[utf8]{inputenc}
\else % if luatex or xelatex
  \ifxetex
    \usepackage{mathspec}
  \else
    \usepackage{fontspec}
  \fi
  \defaultfontfeatures{Ligatures=TeX,Scale=MatchLowercase}
  \newcommand{\euro}{€}
\fi
% use upquote if available, for straight quotes in verbatim environments
\IfFileExists{upquote.sty}{\usepackage{upquote}}{}
% use microtype if available
\IfFileExists{microtype.sty}{%
\usepackage{microtype}
\UseMicrotypeSet[protrusion]{basicmath} % disable protrusion for tt fonts
}{}
\usepackage[margin=1in]{geometry}
\usepackage{hyperref}
\PassOptionsToPackage{usenames,dvipsnames}{color} % color is loaded by hyperref
\hypersetup{unicode=true,
            pdftitle={About Writing Dynamic Documents with R},
            pdfauthor={Author Name1; 1Department of Geography, University of Zurich, Winterthurerstrasse 190, Zurich; name@geo.uzh.ch},
            pdfborder={0 0 0},
            breaklinks=true}
\urlstyle{same}  % don't use monospace font for urls
\usepackage{longtable,booktabs}
\usepackage{graphicx,grffile}
\makeatletter
\def\maxwidth{\ifdim\Gin@nat@width>\linewidth\linewidth\else\Gin@nat@width\fi}
\def\maxheight{\ifdim\Gin@nat@height>\textheight\textheight\else\Gin@nat@height\fi}
\makeatother
% Scale images if necessary, so that they will not overflow the page
% margins by default, and it is still possible to overwrite the defaults
% using explicit options in \includegraphics[width, height, ...]{}
\setkeys{Gin}{width=\maxwidth,height=\maxheight,keepaspectratio}
\setlength{\parindent}{0pt}
\setlength{\parskip}{6pt plus 2pt minus 1pt}
\setlength{\emergencystretch}{3em}  % prevent overfull lines
\providecommand{\tightlist}{%
  \setlength{\itemsep}{0pt}\setlength{\parskip}{0pt}}
\setcounter{secnumdepth}{0}

%%% Use protect on footnotes to avoid problems with footnotes in titles
\let\rmarkdownfootnote\footnote%
\def\footnote{\protect\rmarkdownfootnote}

%%% Change title format to be more compact
\usepackage{titling}

% Create subtitle command for use in maketitle
\newcommand{\subtitle}[1]{
  \posttitle{
    \begin{center}\large#1\end{center}
    }
}

\setlength{\droptitle}{-2em}
  \title{About Writing Dynamic Documents with R}
  \pretitle{\vspace{\droptitle}\centering\huge}
  \posttitle{\par}
  \author{Author Name\textsuperscript{1} \\ \textsuperscript{1}Department of Geography, University of Zurich,
Winterthurerstrasse 190, Zurich \\ \href{mailto:name@geo.uzh.ch}{\nolinkurl{name@geo.uzh.ch}}}
  \preauthor{\centering\large\emph}
  \postauthor{\par}
  \date{}
  \predate{}\postdate{}



% Redefines (sub)paragraphs to behave more like sections
\ifx\paragraph\undefined\else
\let\oldparagraph\paragraph
\renewcommand{\paragraph}[1]{\oldparagraph{#1}\mbox{}}
\fi
\ifx\subparagraph\undefined\else
\let\oldsubparagraph\subparagraph
\renewcommand{\subparagraph}[1]{\oldsubparagraph{#1}\mbox{}}
\fi

\begin{document}
\maketitle
\begin{abstract}
This is the abstract of the template document used to show how to write
publications in R with R Markdown and the help of some packages. Based
on a concrete usecase this document exemplifies some of the caveats that
may occur when writing such document and publish it online on a GIT
repository. It also presents typical usecases in MarkDown usage and
presents some tricks.
\end{abstract}

\subsection{Introduction}\label{introduction}

This example publication is aimed to serve as a motivation on how to
create reproducible documents in R and to advocate in general
reproducible research.

\subsection{State of the Art}\label{state-of-the-art}

Various authors in qualitative and quantitive research argue for that as
many parts of the research workflow reproducible. Brunsdon (2015) state
``Reproducible quantitative research is research that has been
documented sufficiently rigorously that a third party can replicate any
quantitative results that arise''.

To further motivate you, read (Healy 2016,LeVeque et al. (2012),Baker
(2016),Editorial (2016),Pebesma et al. (2012),Vandewalle (2012),Nüst et
al. (2011),Buckheit and Donoho (1995),Healy (2011)) or the short and to
the point editorial of Editorial (2016).

\subsection{Markdown}\label{markdown}

\begin{itemize}
\tightlist
\item
  lists (ordered, unordered)
\item
  figures (figure captions)
\item
  tables
\end{itemize}

\begin{longtable}[c]{@{}ll@{}}
\toprule
Name & Value\tabularnewline
\midrule
\endhead
Reproducible & is coool\tabularnewline
Research & and fun!\tabularnewline
\bottomrule
\end{longtable}

\begin{figure}[htbp]
\centering
\includegraphics{figures/logo.png}
\caption{Reproducible Research Logo}
\end{figure}

\subsection{R Markdown}\label{r-markdown}

\subsubsection{Plots (include figures)}\label{plots-include-figures}

\begin{figure}[htbp]
\centering
\includegraphics{publication-simple_files/figure-latex/googleTrend-1.pdf}
\caption{Timeline of queries for Parc Adula set in the Google search
engine}
\end{figure}

Example to generate and load created image from figure folder

\begin{figure}
\includegraphics[width=0.8\linewidth]{figures/plot} \caption{Plot of the cars data set}\label{fig:map}
\end{figure}

\subsubsection{Data tables}\label{data-tables}

\begin{longtable}[c]{@{}lr@{}}
\caption{Topic mentions.}\tabularnewline
\toprule
Code & Mention\tabularnewline
\midrule
\endfirsthead
\toprule
Code & Mention\tabularnewline
\midrule
\endhead
Biodiversitaet & 5\tabularnewline
Contra Argument & 39\tabularnewline
Pro Argument & 68\tabularnewline
Tourismus allgemein & 48\tabularnewline
\bottomrule
\end{longtable}

\subsection{Discussion and Conclusion}\label{discussion-and-conclusion}

This template based on data of an ongoing presents some typical examples
maybe used in a publication writen in RMarkdown. It presents the
inclusion of data and analysis, features plots, tables, and various
markdown elements and shows how to integrate literature. The generated
files in \emph{PDF}, \emph{Word} or \emph{HTML} often still need fine
some fine-tuning afterwards (particularly in Latex). However, it still
presents a great way documenting the research process, that is easily
shareable and the generation of the initial drafts.

\section{Acknowledgements}\label{acknowledgements}

The Reproducible Research workshop was supported by the InnoPool of the
Department of Geography, University of Zurich.

\section*{References}\label{references}
\addcontentsline{toc}{section}{References}

\hypertarget{refs}{}
\hypertarget{ref-Baker2016}{}
Baker M, 2016, 1,500 scientists lift the lid on reproducibility.
\emph{Nature}. 533(7604):452--454.

\hypertarget{ref-Brunsdon2015}{}
Brunsdon C, 2015, Quantitative methods I: Reproducible research and
quantitative geography. Progress in Human Geography. doi:
\href{https://doi.org/10.1177/0309132515599625}{10.1177/0309132515599625}

\hypertarget{ref-Buckheit1995}{}
Buckheit J, Donoho D, 1995, WaveLab and Reproducible Research.
\emph{Wavelets and Statistics}. 10355--81.

\hypertarget{ref-Nature2016}{}
Editorial, 2016, Reality check on reproducibility. \emph{Nature}.
533(7604):437--437.

\hypertarget{ref-Healy2016}{}
Healy K, 2016, The Plain Person's Guide to Plain Text Social Science.
Healy2016

\hypertarget{ref-Healy2011}{}
Healy K, 2011, Choosing Your Workflow Applications. \emph{The Political
Methodologist}. 18(2):9--18.

\hypertarget{ref-Leveque2012}{}
LeVeque RJ, Mitchell IM, Stodden V, 2012, Reproducible research for
scientific computing: Tools and strategies for changing the culture.
\emph{Computing in Science \& Engineering}. 14(4):13--17.

\hypertarget{ref-Nuest2011}{}
Nüst D, Stasch C, Pebesma E, 2011, Connecting R to the sensor Web. In:
Lecture notes in geoinformation and cartography. 227--246

\hypertarget{ref-Pebesma2012}{}
Pebesma E, Nüst D, Bivand R, 2012, The R software environment in
reproducible geoscientific research. \emph{Eos, Transactions American
Geophysical Union}. 93(16):163--163.

\hypertarget{ref-Vandewalle2012}{}
Vandewalle P, 2012, Code Sharing Is Associated with Research Impact in
Image Processing. \emph{Computing in Science \& Engineering}.
14(4):42--47.

\end{document}
